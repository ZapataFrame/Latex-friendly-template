\documentclass[12pt]{article}
\usepackage{nord-style}
\usepackage{lipsum}

% Document Metadata
\newcommand{\mytitle}{Manual de Uso de la Plantilla}
\newcommand{\mysubtitle}{Edición Especial: Pingüinos de Madagascar}
\newcommand{\mydoctype}{GUÍA DE REFERENCIA}
\newcommand{\myauthor}{Kowalski}
\newcommand{\myprofessor}{Skipper}
\newcommand{\mydate}{Antártida, \today}

\renewcommand{\authorlabel}{ESTRATEGA}

\begin{document}

\makecustomcover{\mytitle}{\mysubtitle}{\myauthor}{\myprofessor}{\mydate}{\mydoctype}

\clearpage
\setheadercolor{\toccolor}
\tableofcontents
\clearpage
\setheadercolor{\themecolor}

\section{Introducción}
Este documento sirve como una guía exhaustiva para la utilización de la plantilla LaTeX "Nord Theme". A continuación, se presentan ejemplos prácticos de los elementos más comunes, todos explicados con la ayuda de nuestros amigos los pingüinos.

% ==========================================
%  1. Tablas
% ==========================================
\section{Tablas}
Las tablas son fundamentales para organizar datos de misiones y raciones de pescado.

\subsection{Tabla Simple}
Una tabla estándar utilizando el paquete \texttt{booktabs} para un acabado profesional.

% latex tablasimple
\begin{table}[h]
    \centering
    \caption{Inventario de Pescado}
    \label{tab:simple}
    \begin{tabular}{|l|l|r|}
        \hline
        \tableheadersecondary
        \hc{Especie} & \hc{Origen} & \hc{Cantidad (kg)} \\
        \hline
        Arenque & Atlántico Norte & 500 \\
        \hline
        Sardina & Mediterráneo & 350 \\
        \hline
        Krill & Antártico & 1200 \\
        \hline
    \end{tabular}
\end{table}
% //===========

\subsection{Tabla Stripeada}
Utiliza colores alternos para facilitar la lectura de grandes volúmenes de datos.

% latex tablastripeada
\begin{table}[h]
    \centering
    \rowcolors{2}{nord6}{white} % Alternar colores desde la fila 2
    \caption{Registro de Actividades Diarias}
    \label{tab:striped}
    \begin{tabular}{|c|l|l|}
        \hline
        \tableheaderprimary
        \hc{Hora} & \hc{Agente} & \hc{Actividad} \\
        \hline
        06:00 & Skipper & Entrenamiento matutino \\
        07:00 & Kowalski & Análisis de datos \\
        08:00 & Rico & Demoliciones controladas \\
        09:00 & Cabo & Avistamiento de mariposas \\
        10:00 & Skipper & Reunión estratégica \\
        \hline
    \end{tabular}
\end{table}
% //===========



\subsection{Tabla de Ancho Completo}
Utiliza \texttt{tabularx} para ocupar todo el ancho de la página automáticamente.

% latex tablafull
\begin{table}[h]
    \centering
    \caption{Distribución de Recursos (Ancho Completo)}
    \label{tab:fullwidth}
    \begin{tabularx}{\textwidth}{|l|X|r|}
        \hline
        \tableheadersecondary
        \hc{Recurso} & \hc{Descripción Detallada} & \hc{Prioridad} \\
        \hline
        Pescado & Suministro vital de energía para operaciones encubiertas y sobornos a focas. & Alta \\
        \hline
        Cinta Adhesiva & Herramienta universal para reparaciones, trampas y silenciamiento de testigos. & Media \\
        \hline
        Dinamita & Solución para puertas cerradas y problemas persistentes. & Crítica \\
        \hline
    \end{tabularx}
\end{table}
% //===========

\subsection{Tablas con Encabezados de Color}
Estilos personalizados para resaltar la cabecera.

% latex tablaheader1
\begin{table}[h]
    \centering
    \caption{Personal de Misión (Header Primario)}
    \label{tab:header1}
    \begin{tabular}{|l|l|r|}
        \hline
        \tableheaderprimary
        \hc{Nombre} & \hc{Rango} & \hc{Especialidad} \\
        \hline
        Skipper & Comandante & Liderazgo \\
        \hline
        Kowalski & Teniente & Estrategia \\
        \hline
        Rico & Sargento & Demoliciones \\
        \hline
    \end{tabular}
\end{table}
% //===========

% latex tablaheader2
\begin{table}[h]
    \centering
    \caption{Objetivos Secundarios (Header Secundario)}
    \label{tab:header2}
    \begin{tabular}{|l|l|r|}
        \hline
        \tableheadersecondary
        \hc{Objetivo} & \hc{Ubicación} & \hc{Dificultad} \\
        \hline
        Robar Cheezy Dibbles & Tienda de Regalos & Baja \\
        \hline
        Hackear TV & Oficina de Alice & Media \\
        \hline
    \end{tabular}
\end{table}
% //===========

% ==========================================
%  2. Listas
% ==========================================
\section{Listas}
Organización de planes y pasos a seguir.

\subsection{Listas No Ordenadas}
Para elementos sin orden específico.

% latex listaitem
\begin{itemize}
    \item Operación "Lindo y Gordito".
    \item Plan de escape del zoológico.
    \item Búsqueda de snacks con sabor a queso.
\end{itemize}
% //===========

\subsection{Listas Ordenadas}
Para secuencias de pasos.

% latex listaenum
\begin{enumerate}
    \item Infiltrarse en el casino.
    \item Hackear el sistema de seguridad.
    \item Extraer el paquete.
    \item Escapar en un auto deportivo.
\end{enumerate}
% //===========

\subsection{Listas de Definición}
Para glosarios o descripciones de roles.

% latex listadesc
\begin{description}
    \item[Skipper] El líder intrépido y paranoico.
    \item[Kowalski] El cerebro y analista científico.
    \item[Rico] El especialista en armas y demoliciones.
    \item[Cabo] El corazón del equipo y cebo adorable.
\end{description}
% //===========

\section{Estructura de Columnas}
Organización del texto en múltiples columnas para boletines o comparaciones.

\subsection{Dos Columnas (1/2)}
\begin{multicols}{2}
    \textbf{Columna Izquierda:} \\
    El pingüino emperador es la más grande de todas las especies de pingüinos. Pueden superar los 1,20 m de altura y pesan entre 20 y 45 kg.
    
    \columnbreak
    
    \textbf{Columna Derecha:} \\
    Viven en la Antártida y son famosos por sus viajes heroicos para reproducirse en el invierno antártico.
\end{multicols}

\subsection{Tres Columnas (1/3)}
\begin{multicols}{3}
    \textbf{Paso 1:} \\
    Identificar el objetivo visualmente.
    
    \columnbreak
    
    \textbf{Paso 2:} \\
    Calcular la trayectoria de deslizamiento.
    
    \columnbreak
    
    \textbf{Paso 3:} \\
    Ejecutar la maniobra con estilo.
\end{multicols}
% //===========

% ==========================================
%  3. Figuras
% ==========================================
\section{Figuras}
Ilustraciones tácticas y visualización de objetivos.

\subsection{Figura Grande}
Una imagen centrada que ocupa un ancho considerable.

% latex figuragrande
\begin{figure}[H]
    \centering
    \includegraphics[width=0.7\textwidth]{img/penguin.png}
    \caption{Sujeto de prueba estándar (Vista Frontal).}
    \label{fig:grande}
\end{figure}
% //===========

\subsection{Figura Chica}
Ideal para logotipos o imágenes pequeñas.

% latex figurachica
\begin{figure}[H]
    \centering
    \includegraphics[width=0.3\textwidth]{img/penguin.png}
    \caption{Versión compacta para transporte encubierto.}
    \label{fig:chica}
\end{figure}
% //===========

\subsection{Figuras Múltiples}
Comparativa lado a lado utilizando subfiguras.

% latex figuramultiple
\begin{figure}[H]
    \centering
    \begin{subfigure}[b]{0.45\textwidth}
        \centering
        \includegraphics[width=\textwidth]{img/penguin.png}
        \caption{Antes del entrenamiento.}
        \label{fig:antes}
    \end{subfigure}
    \hfill
    \begin{subfigure}[b]{0.45\textwidth}
        \centering
        \includegraphics[width=\textwidth]{img/penguin.png}
        \caption{Después del entrenamiento.}
        \label{fig:despues}
    \end{subfigure}
    \caption{Evolución del recluta tras el programa intensivo.}
    \label{fig:multiple}
\end{figure}
% //===========

\subsection{Diseño Complejo de Figuras}
Una figura grande a la izquierda y tres pequeñas a la derecha.

% latex figuracompleja
\begin{figure}[H]
    \centering
    % Left Column: Large Image (1:1 aspect ratio implied by sketch)
    \begin{minipage}{0.6\textwidth}
        \centering
        \includegraphics[width=\linewidth]{img/penguin.png}
    \end{minipage}\hfill
    % Right Column: Vertical Stack
    \begin{minipage}{0.35\textwidth}
        \centering
        \includegraphics[width=0.8\linewidth]{img/penguin.png}
        \vspace{1em} % Space between images
        
        \includegraphics[width=0.8\linewidth]{img/penguin.png}
        \vspace{1em}
        
        \includegraphics[width=0.8\linewidth]{img/penguin.png}
    \end{minipage}
    \caption{Vista General del Cuartel y Detalles Tácticos (A, B y C).}
    \label{fig:compleja}
\end{figure}
% //===========

% ==========================================
%  4. Código
% ==========================================
\section{Código}
Fragmentos de algoritmos utilizados en la "Kowalski Analysis Machine".

\subsection{Bloque de Código}
Ejemplo de un script en Python para calcular la trayectoria de vuelo (teórica).

% latex codigobloque
\begin{lstlisting}[language=Python, caption=CalculoTrayectoria.py]
def calcular_vuelo(peso, envergadura):
    """
    Calcula si un pingüino puede volar.
    Spoiler: No puede.
    """
    gravedad = 9.81
    sustentacion = 0
    
    if peso > 0 and envergadura > 0:
        print("Iniciando secuencia de despegue...")
        # Aplicar física de dibujos animados
        sustentacion = (envergadura * 100) / peso
        
    if sustentacion > gravedad:
        return True
    else:
        return "Solo si te lanzan de un cañón"
\end{lstlisting}
% //===========

\subsection{Bloque de Código (Sin Números)}
Para snippets rápidos donde la línea no importa.

% latex codigononum
\begin{lstlisting}[style=nordcode-nonum, language=Bash, caption=InstalarDependencias.sh]
sudo apt-get update
sudo apt-get install libpenguin-dev
\end{lstlisting}
% //===========

\subsection{Bloque de Código (Estilo Oscuro)}
Para resaltar bloques críticos o de otro contexto.

% latex codigodark
\begin{lstlisting}[style=nordcode-dark, language=Python, caption=NucleoDelSistema.py]
class SelfDestruct:
    def __init__(self):
        self.timer = 10
        
    def activate(self):
        while self.timer > 0:
            print(f"Autodestrucción en {self.timer}...")
            self.timer -= 1
        print("KABOOM!")
\end{lstlisting}
% //===========

\subsection{Código en Línea}
Para mencionar variables o comandos específicos.

% latex codigolinea
El comando \texttt{sudo apt-get install fish} es esencial para mantener la moral alta. La clase \texttt{Penguin} hereda de \texttt{Bird} pero sobreescribe el método \texttt{fly()} con una excepción de tipo \texttt{NotImplementedError}.
% //===========

% ==========================================
%  5. Tipos de Texto
% ==========================================
\section{Tipos de Texto}
Diferentes formas de enfatizar la información en los reportes de misión.

\subsection{Estilos Básicos}
% latex textoestilos
Texto \textbf{en negrita} para órdenes directas. \\
Texto \textit{en cursiva} para pensamientos internos o sarcasmo. \\
Texto \underline{subrayado} para énfasis crítico. \\
Texto \texttt{monoespaciado} para datos de teletipo.
% //===========

\subsection{Citas}
Para referencias a frases célebres del equipo.

% latex textocita
\begin{quote}
    "¡Sonrían y saluden, muchachos! ¡Sonrían y saluden!"
    \flushright --- Skipper
\end{quote}
% //===========

\subsection{Texto Resaltado (Colores Nord)}
Uso de la paleta de colores para categorizar información.

% latex textocolor
\textcolor{nord11}{\textbf{ALERTA ROJA:}} Foca leopardo detectada en el sector 7. \\
\textcolor{nord13}{\textbf{ADVERTENCIA:}} Nivel de snacks bajo mínimos. \\
\textcolor{nord14}{\textbf{ESTADO:}} Operación exitosa. \\
\textcolor{nord10}{\textbf{Información:}} La temperatura del agua es óptima (-2°C).
% //===========

\end{document}
